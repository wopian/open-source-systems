% !TeX root = ../index.tex
\chapter{Week 1}
\fxnote{Add output images}

\section{Exercise 1}

\subfile{pyg/src/1-week-1/ex1}
\captionsetup{type=figure}\captionof{figure}{\url{http://intweb.bucks.ac.uk/~21606555/oos/1-week-1/ex1.php}}

\section{Exercise 2}

\subfile{pyg/src/1-week-1/ex2}
\captionsetup{type=figure}\captionof{figure}{\url{http://intweb.bucks.ac.uk/~21606555/oos/1-week-1/ex2.php}}

\section{Exercise 3}

\subfile{pyg/src/1-week-1/ex3}
\captionsetup{type=figure}\captionof{figure}{\url{http://intweb.bucks.ac.uk/~21606555/oos/1-week-1/ex3.php}}

\section{Exercise 4}

The last argument of \texttt{gmddate()\texttt} allows you to provide a specific date to format. This argument is optional and will use the current date and time if it is not provided.

\subfile{pyg/src/1-week-1/ex4}
\captionsetup{type=figure}\captionof{figure}{\url{http://intweb.bucks.ac.uk/~21606555/oos/1-week-1/ex4.php}}

\section{Exercise 5}

\subfile{pyg/src/1-week-1/ex5}
\captionsetup{type=figure}\captionof{figure}{\url{http://intweb.bucks.ac.uk/~21606555/oos/1-week-1/ex5.php}}

\section{Exercise 6}

\subfile{pyg/src/1-week-1/ex6}
\captionsetup{type=figure}\captionof{figure}{\url{http://intweb.bucks.ac.uk/~21606555/oos/1-week-1/ex6.php}}

\section{Exercise 7}

\subfile{pyg/src/1-week-1/ex7}
\captionsetup{type=figure}\captionof{figure}{\url{http://intweb.bucks.ac.uk/~21606555/oos/1-week-1/ex7.php}}
