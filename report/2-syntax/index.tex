% !TeX root = ../index.tex
\chapter{Syntax}
\fxnote{Add output images}

\section{Exercise 1}

Converted the shorthand PHP tags to \texttt{<?php} as IntWeb has disabled the language feature.

\url{http://intweb.bucks.ac.uk/~21606555/oos/2-syntax/ex1.php}
\subfile{pyg/src/2-syntax/ex1}
\captionsetup{type=figure}\captionof{figure}{ex1.php}

\section{Exercise 2}

\url{http://intweb.bucks.ac.uk/~21606555/oos/2-syntax/ex2.php}
\subfile{pyg/src/2-syntax/ex2}
\captionsetup{type=figure}\captionof{figure}{ex2.php}

\section{Exercise 3}

\url{http://intweb.bucks.ac.uk/~21606555/oos/2-syntax/ex3.php}
\subfile{pyg/src/2-syntax/ex3}
\captionsetup{type=figure}\captionof{figure}{ex3.php}

\section{Exercise 4}

\url{http://intweb.bucks.ac.uk/~21606555/oos/2-syntax/ex4.php}
\subfile{pyg/src/2-syntax/ex4}
\captionsetup{type=figure}\captionof{figure}{ex4.php}

\section{Exercise 5}

\url{http://intweb.bucks.ac.uk/~21606555/oos/2-syntax/ex5.php}
\subfile{pyg/src/2-syntax/ex5}
\captionsetup{type=figure}\captionof{figure}{ex5.php}

\section{Exercise 6}

If an index position isn't specified when assigning a value to an array then the value is inserted after the last index position in the array. In this case, $76$ is inserted into the 4th index.\\

\url{http://intweb.bucks.ac.uk/~21606555/oos/2-syntax/ex6.php}
\subfile{pyg/src/2-syntax/ex6}
\captionsetup{type=figure}\captionof{figure}{ex6.php}

\section{Exercise 7}

11 lines will be displayed as it will loop from 0 until it reaches 10.\\

\url{http://intweb.bucks.ac.uk/~21606555/oos/2-syntax/ex7.php}
\subfile{pyg/src/2-syntax/ex7}
\captionsetup{type=figure}\captionof{figure}{ex7.php}

\section{Exercise 8}

Fixed the for loop syntax where a comma was used instead of a semi-colon.\\

\url{http://intweb.bucks.ac.uk/~21606555/oos/2-syntax/ex8.php}
\subfile{pyg/src/2-syntax/ex8}
\captionsetup{type=figure}\captionof{figure}{ex8.php}

\section{Exercise 9}

\url{http://intweb.bucks.ac.uk/~21606555/oos/2-syntax/ex9.php}
\subfile{pyg/src/2-syntax/ex9}
\captionsetup{type=figure}\captionof{figure}{ex9.php}

\section{Exercise 10}

This has been revised to be displayed in a table from the exercise 12 task.\\

\url{http://intweb.bucks.ac.uk/~21606555/oos/2-syntax/ex10.php}
\subfile{pyg/src/2-syntax/ex10}
\captionsetup{type=figure}\captionof{figure}{ex10.php}

\section{Exercise 11}

I opted to use the \texttt{sizeof} function to calculate the averages so that the average will always be correct if marks are removed or added to the array.\\

\url{http://intweb.bucks.ac.uk/~21606555/oos/2-syntax/ex11.php}
\subfile{pyg/src/2-syntax/ex11}
\captionsetup{type=figure}\captionof{figure}{ex11.php}

\section{Exercise 12}

\url{http://intweb.bucks.ac.uk/~21606555/oos/2-syntax/ex12.php}
\subfile{pyg/src/2-syntax/ex12}
\captionsetup{type=figure}\captionof{figure}{ex12.php}
