% !TeX root = ../index.tex
\chapter{Syntax}

\section{Exercise 1}

Converted the shorthand PHP tags to \texttt{<?php} as IntWeb has disabled the language feature.

\subfile{pyg/src/2-syntax/ex1}
\captionsetup{type=figure}\captionof{figure}{\url{http://intweb.bucks.ac.uk/~21606555/oos/2-syntax/ex1.php}}

\section{Exercise 2}

\subfile{pyg/src/2-syntax/ex2}
\captionsetup{type=figure}\captionof{figure}{\url{http://intweb.bucks.ac.uk/~21606555/oos/2-syntax/ex2.php}}

\section{Exercise 3}

\subfile{pyg/src/2-syntax/ex3}
\captionsetup{type=figure}\captionof{figure}{\url{http://intweb.bucks.ac.uk/~21606555/oos/2-syntax/ex3.php}}

\section{Exercise 4}

\subfile{pyg/src/2-syntax/ex4}
\captionsetup{type=figure}\captionof{figure}{\url{http://intweb.bucks.ac.uk/~21606555/oos/2-syntax/ex4.php}}

\section{Exercise 5}

\subfile{pyg/src/2-syntax/ex5}
\captionsetup{type=figure}\captionof{figure}{\url{http://intweb.bucks.ac.uk/~21606555/oos/2-syntax/ex5.php}}

\section{Exercise 6}

If an index position isn't specified when assigning a value to an array then the value is inserted after the last index position in the array. In this case, $76$ is inserted into the 4th index.

\subfile{pyg/src/2-syntax/ex6}
\captionsetup{type=figure}\captionof{figure}{\url{http://intweb.bucks.ac.uk/~21606555/oos/2-syntax/ex6.php}}

\section{Exercise 7}

11 lines will be displayed as it will loop from 0 until it reaches 10.

\subfile{pyg/src/2-syntax/ex7}
\captionsetup{type=figure}\captionof{figure}{\url{http://intweb.bucks.ac.uk/~21606555/oos/2-syntax/ex7.php}}

\section{Exercise 8}

Fixed the for loop syntax where a comma was used instead of a semi-colon.

\subfile{pyg/src/2-syntax/ex8}
\captionsetup{type=figure}\captionof{figure}{\url{http://intweb.bucks.ac.uk/~21606555/oos/2-syntax/ex8.php}}

\section{Exercise 9}

\subfile{pyg/src/2-syntax/ex9}
\captionsetup{type=figure}\captionof{figure}{\url{http://intweb.bucks.ac.uk/~21606555/oos/2-syntax/ex9.php}}

\section{Exercise 10}

This has been revised to be displayed in a table from the exercise 12 task.

\subfile{pyg/src/2-syntax/ex10}
\captionsetup{type=figure}\captionof{figure}{\url{http://intweb.bucks.ac.uk/~21606555/oos/2-syntax/ex10.php}}

\section{Exercise 11}

I opted to use the \texttt{sizeof} function to calculate the averages so that the average will always be correct if marks are removed or added to the array.

\subfile{pyg/src/2-syntax/ex11}
\captionsetup{type=figure}\captionof{figure}{\url{http://intweb.bucks.ac.uk/~21606555/oos/2-syntax/ex11.php}}

\section{Exercise 12}

\subfile{pyg/src/2-syntax/ex12}
\captionsetup{type=figure}\captionof{figure}{\url{http://intweb.bucks.ac.uk/~21606555/oos/2-syntax/ex12.php}}
